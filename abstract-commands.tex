% command for a new time slot
\newcommand{\talkTime}{9:99}
\newcommand{\newTimeslot}[1]{\newpage\renewcommand{\talkTime}{#1}}

% new time slot but without a pagebreak
\newcommand{\newSmallTimeslot}[1]{\renewcommand{\talkTime}{#1}}

% initialise \conferenceDay 
\newcommand{\conferenceDay}{Noday}

% define colours
\definecolor{beige-uni}{cmyk}{0.05 0.05 0.09 0.00}
%\definecolor{beige-uni}{cmyk}{0.00 0.17 0.15 0.34}
\definecolor{beige}{cmyk}{0 0.02 0.06 0.07}
\definecolor{rose}{cmyk}{0 0.25 0.39 0}
\colorlet{GHS}{beige-uni}
\colorlet{HSO}{beige-uni}
\colorlet{HSW}{beige-uni}
\colorlet{academic}{beige-uni}
\colorlet{KHS}{beige-uni}
\colorlet{Mathematikon C}{beige-uni}
%\definecolor{GHS}{cmyk}{0.16 0 .05 0}
%\definecolor{HSO}{cmyk}{0.02 0 0.12 0.11}
%\definecolor{academic}{cmyk}{0 0.02 0.23 0}
%\definecolor{HSW}{cmyk}{0.35 0 0.33 0.16}
%\definecolor{KHS}{cmyk}{0 .24 0.29 .04}
%\definecolor{Mathematikon C}{cmyk}{.16 0 0.35 0}
\definecolor{ochsenblutrot}{cmyk}{0.30 1.0 0.90 0.65}
\colorlet{table-header}{beige-uni}

\input{wallpaper/day-layers.tex}

% define page style for cutting marks without anything else
\DeclareNewLayer[background, oddorevenpage, width=125mm,%
height=169mm, contents={%
  \begin{tikzpicture}[x=1mm, y=1mm]
    \drawCropMarks
  \end{tikzpicture}
}]{cropmarksplain}

% define default page style (cutting marks with page number)
\DeclareNewLayer[background, oddorevenpage, width=125mm,%
height=169mm, contents={%
  \begin{tikzpicture}[x=1mm, y=1mm]
    \drawCropMarks
  \end{tikzpicture}
}]{cropmarksevery}
\newpairofpagestyles[scrheadings]{cropmarksstyle}{}
\AddLayersAtBeginOfPageStyle{cropmarksstyle}{cropmarksevery}

% page style for title pages
\DeclareNewLayer[background, oddorevenpage, width=125mm,%
height=169mm, contents={%
  \includegraphics{wallpaper/front-cover-with-crop-marks.pdf}%
}]{titlelayer}
\newpairofpagestyles[]{titlestyle}{}
\AddLayersAtBeginOfPageStyle{titlestyle}{titlelayer}

% define alias commands for all three days
\def\saturday{Saturday}
\def\sunday{Sunday}
\def\monday{Monday}

% define Saturday page style
\newpairofpagestyles[scrheadings]{saturday-table}{}
\AddLayersAtBeginOfPageStyle{saturday-table}{saturdayeven}
\AddLayersAtBeginOfPageStyle{saturday-table}{saturdayoddrotated}
\newpairofpagestyles[scrheadings]{saturday}{}
\AddLayersAtBeginOfPageStyle{saturday}{saturdayeven}
\AddLayersAtBeginOfPageStyle{saturday}{saturdayodd}

% define Sunday page style
\newpairofpagestyles[scrheadings]{sunday-table}{}
\AddLayersAtBeginOfPageStyle{sunday-table}{sundayeven}
\AddLayersAtBeginOfPageStyle{sunday-table}{sundayoddrotated}
\newpairofpagestyles[scrheadings]{sunday}{}
\AddLayersAtBeginOfPageStyle{sunday}{sundayeven}
\AddLayersAtBeginOfPageStyle{sunday}{sundayodd}

% define Monday page style
\newpairofpagestyles[scrheadings]{monday-table}{}
\AddLayersAtBeginOfPageStyle{monday-table}{mondayeven}
\AddLayersAtBeginOfPageStyle{monday-table}{mondayoddrotated}
\newpairofpagestyles[scrheadings]{monday}{}
\AddLayersAtBeginOfPageStyle{monday}{mondayeven}
\AddLayersAtBeginOfPageStyle{monday}{mondayodd}

% \setpagebackground selects the page style to be used depending on the current day. Each day has
% its own page style.
\newcommand{\setPageBackground}{ %
  \ifthenelse{\equal{\conferenceDay}{\saturday}}{%
    \pagestyle{saturday}
  }{}
  \ifthenelse{\equal{\conferenceDay}{\sunday}}{%
    \pagestyle{sunday}
  }{}
  \ifthenelse{\equal{\conferenceDay}{\monday}}{%
    \pagestyle{monday}
  }{}
}


% additional column type for tables
\newcolumntype{Y}[1]{>{\RaggedRight\arraybackslash}p{#1}}

%% length of the title boxes
\newlength{\titleboxwidth}
\advance\titleboxwidth by -6pt

\newlength{\roomWidth}
\newlength{\timeWidth}
\newlength{\roomTimeWidth}
\newlength{\titleWidth}
\newcommand{\tmpRoomTimeWidth}{}
\newcommand{\tmpTitleWidth}{}

\newcommand{\setAbstract}[6]{%
  % 1. speaker
  % 2. title
  % 3. subtitle
  % 4. abstract (Text)
  % 5. colour
  % 6. room
  \setPageBackground%
  \calculateRoomTimeAndTitleWidth{#1}{#6}%
  \noindent\fcolorbox{white}{#5}{%
    \parbox{\titleboxwidth}{%
      \isSpeakerEmpty{#1}{#2}{#6}%
      \isSubtitleEmpty{#3}%
    }%
  }%
  %
  \isAbstractEmpty{#4}%
  \vspace{0.5em}% space to the next talk even if there is no abstract
}

% Calculate width of title and room/time field
% Arguments: speaker, room
\makeatletter
  \newcommand{\calculateRoomTimeAndTitleWidth}[2]{%
    \@ifmtarg{#1}{% speaker is empty
      \settowidth{\roomWidth}{#2, \talkTime}%
    }{% speaker is not empty
      \settowidth{\roomWidth}{#2}%
    }%
    \settowidth{\timeWidth}{\talkTime}%
    \directlua{require("lua/titleWidth")}%
    \setlength{\titleboxwidth}{\directlua{getTitleBoxWidth()}}%
    \renewcommand{\tmpRoomTimeWidth}{\directlua{getTimeRoomWidth()}}%
    \setlength{\roomTimeWidth}{\tmpRoomTimeWidth}%
    \renewcommand{\tmpTitleWidth}{\directlua{getTitleWidth()}}%
    \setlength{\titleWidth}{\tmpTitleWidth}%
  }%
\makeatother

% Lay out the subtitle
% has to be a separate function and has to be surrounded by \makeatletter for technical reasons
\makeatletter
  \newcommand{\isSubtitleEmpty}[1]{%
    \@ifnotmtarg{#1}{%
      \par
      \RaggedRight
      \vspace{0.3\baselineskip}
      \noindent%
      \bfseries%
      #1%
    }
  }
\makeatother

% lay out the speaker if there is any
% We assume that there is only a subtitle if the talk has a speaker.
\makeatletter
  \newcommand{\isSpeakerEmpty}[3]{%
    % Arguments:
    % 1. speaker
    % 2. title
    % 3. room
    \@ifmtarg{#1}{%
      \begin{minipage}[t][][t]{\titleWidth}
        \RaggedRight
        \noindent%
        {\large \sectfont #2}%
      \end{minipage}%
      \hfill
      \begin{minipage}[t][][t]{\roomTimeWidth}
        \RaggedLeft%
        \noindent #3, \talkTime%
      \end{minipage}%
    }%
    {%
      \begin{minipage}[t][][t]{\titleWidth}
        \RaggedRight
        \emph{#1} % speaker
        \par%
        \vspace{0.3\baselineskip}
        \noindent%
        {\large \sectfont #2}%
      \end{minipage}%
      \hfill
      \begin{minipage}[t][][t]{\roomTimeWidth}
        \RaggedLeft
        \talkTime%
        \par%
        \noindent #3% room
      \end{minipage}%
    }%
  }
\makeatother

% Lay out the abstract if there is any
% has to be a separate function and has to be surrounded by \makeatletter for technical reasons
\makeatletter
\newcommand{\isAbstractEmpty}[1]{%
  \ifstrempty{#1}{%
    \vspace{1.5em}%
  }{%
    \vspace{0.5em}\newline%
    #1 \par% % abstract
    \vspace{1.5em}% space to the next talk even if there is an abstract
  }%
}
\makeatother

% session at GHS
\newcommand{\abstractGHS}[4]%
{%
  \setAbstract{#1}{#2}{#3}{#4}{beige-uni}{Großer Hörsaal}
}

% abstract at HSO 
\newcommand{\abstractHSO}[4]%
{%
  \setAbstract{#1}{#2}{#3}{#4}{beige-uni}{Hörsaal Ost}
}

% abstract for academic talks
\newcommand{\abstractAcademic}[4]%
{%
  \setAbstract{#1}{#2}{#3}{#4}{beige-uni}{Hörsaal West}
}

% abstract at HSW
\newcommand{\abstractHSW}[4]%
{%
  \setAbstract{#1}{#2}{#3}{#4}{beige-uni}{Hörsaal West}
}

% abstract at KHS
\newcommand{\abstractKHS}[4]%
{%
  \setAbstract{#1}{#2}{#3}{#4}{beige-uni}{Kleiner Hörsaal}
}

% abstract at Mathematikon C
\newcommand{\abstractMathematikonC}[4]%
{%
  \setAbstract{#1}{#2}{#3}{#4}{beige-uni}{Mathematikon C}
}

% abstract at a different location
\newcommand{\abstractOther}[5]%
{%
  \setAbstract{#1}{#2}{#3}{#4}{beige-uni}{#5}
}

% infobox for workshops (they don't have an abstract in the booklet)
\newcommand{\workshopbox}[3]{%
  % 1. titel
  % 2. speaker
  % 3. Room
  \setlength\tabcolsep{0pt}
  \noindent\fcolorbox{white}{dezentrot}{%
    \parbox{\titleboxwidth}{%
      \noindent%
      \begin{tabu}{X[5L]r}
        \emph{#2} % Sprecher
        &
        \talkTime
        \tabularnewline
        {\noindent\large \bfseries #1}% % title
        &
        #3
        \tabularnewline
      \end{tabu}
    }
  }
  \setlength\tabcolsep{6pt} % set column padding back to default
}

% too long
\newcommand{\tooLong}{Dieser Text ist viel zu lang. Dieser Text ist viel zu lang. Dieser Text ist viel zu lang. Dieser Text ist viel zu lang. Dieser Text ist viel zu lang. Dieser Text ist viel zu lang. Dieser Text ist viel zu lang. Dieser Text ist viel zu lang. Dieser Text ist viel zu lang. Dieser Text ist viel zu lang. Dieser Text ist viel zu lang. Dieser Text ist viel zu lang. Dieser Text ist viel zu lang. Dieser Text ist viel zu lang. }

\newlength{\fboxwidth}

\def\workshopsSection{workshopsSection}
\def\abstractsSection{abstractsSection}

% boxes for text-only advertisement texts by our sponsors
\newcommand{\sponsorBox}[4]{%
  \setlength{\fboxwidth}{\textwidth}
  \advance\fboxwidth by -7.0pt
  \abstractSponsorbox{#1}{#2}{#3}{#4}{\workshopsSection}%
}

\newcommand{\sponsorBoxA}[4]{%
  \setlength{\fboxwidth}{\textwidth}
  \advance\fboxwidth by -10.0pt
  \abstractSponsorbox{#1}{#2}{#3}{#4}{\abstractsSection}%
}

\newcommand{\sponsorBoxLandscape}[4]{%
  \setlength{\fboxwidth}{\linewidth}
  \advance\fboxwidth by -10.0pt
  \abstractSponsorbox{#1}{#2}{#3}{#4}{\abstractsSection}%
}

%% store \parindent in separate variable because it is resetted to 0 in parboxes
\newlength{\saveparindent}
\setlength{\saveparindent}{\parindent}

%% box for advertisment by a sponsor
%% 1. logo (leave empty if none)
%% 2. width of the logo (leave empty if none
%% 3. number of required lines of the logo (due to usage of wrapfigure)
%% 4. text
%% 5. where we are in the booklet (\workshopsSection or \abstractsSection}
\makeatletter
  \newcommand{\abstractSponsorbox}[5]{%
    \setlength{\fboxsep}{4.5pt}%
    \noindent%
    \ifthenelse{\equal{#5}{\workshopsSection}}{%
      \hspace{2.65pt}%
    }{%
      \hspace{-1pt}%
    }%
    \fcolorbox{gray}{white}{%
      \parbox{\fboxwidth}{
        \setlength{\parindent}{\saveparindent}%
        \@ifmtarg{#1}{}{%
          \begin{wrapfigure}[#3]{r}[0pt]{#2}
            \centering\vspace{-1\baselineskip}
            \includegraphics[width=#2]{#1}
          \end{wrapfigure}
        }
  
        \noindent #4
      }%
    }
    \setlength{\fboxsep}{3pt}
  }
  \setlength{\fboxsep}{3pt}
\makeatother

% definition of column types for the schedule tables
\newcolumntype{Z}[1]{>{\RaggedRight\arraybackslash}p{#1}}%
\newcolumntype{C}[1]{>{\Centering\arraybackslash}p{#1}}%

% common implementation of typesetting of a session in the tables
\newcommand{\talkInternal}[2]{%
  \textbf{#1}
  \ifthenelse{\equal{#2}{}}{}{%
    \newline\emph{#2}%
  }
}

% common implementation of typesetting of a session in the tables -- singleline mode
\newcommand{\talkInternalSingleLine}[2]{%
  \textbf{#1}
  \hspace{0.5em}
  \ifthenelse{\equal{#2}{}}{}{%
    \emph{#2}%
  }
}

% macro to typeset a talk in the schedule tables spanning over more than one row:
% usage: \longTalk{rowcount}{title}{speaker}
\newcommand{\longTalk}[3]{%
  &
  \multirow{#1}{\linewidth}{%
    \parbox{\linewidth}{
      %HACK Inserting a \vspace here is a dirty hack but it works.
      \vspace{0.45\baselineskip}
      \talkInternal{#2}{#3}%
    }
  }%
}%

% macro to typeset a talk in the schedule tables:
% usage: \talk{title}{speaker}
\newcommand{\talk}[2]{%
  &
  \talkInternal{#1}{#2}%
}%

% macro to typeset a talk in the schedule tables, single line mode:
% usage: \talkSingleLine{title}{speaker}
\newcommand{\talkSingleLine}[2]{%
  &
  \talkInternalSingleLine{#1}{#2}%
}%


% macro to typeset a talk in the schedule tables spanning over more than one column:
% usage: \multiColTalk{columns}{columnSpecs}{title}{speaker}
\newcommand{\multiColTalk}[4]{%
  &
  \multicolumn{#1}{#2}{\talkInternal{#3}{#4}}%
}%

% set time in scheduel table
\newcommand{\tableRowFirstCell}[1]{%
  \cellcolor{table-header}
  \textcolor{black}{#1}%
}

% set time in scheduel table
\newcommand{\tableColHead}[1]{%
  \cellcolor{table-header}%
  \textcolor{black}{#1}%
}

\newcommand{\workshop}[3]%
{%
  \workshopbox{#1}{#2}{#3}
}%

\newcommand{\otherevent}[1]%
{%
  & \textbf{#1}
}%

\newcommand{\audimaxEvent}[2]%
{%
  &
  \multicolumn{3}{c}{
    \textbf{#1} (Audimax) \par \emph{#2}
  }
}%

\newcommand{\coffeespace}{\vspace{0.4em}}
\newcommand{\workshopspace}{\vspace{0.5em}\\}

% define colors
\definecolor{commongray}{gray}{.9}
\definecolor{tableRuleGray}{gray}{.9}
\definecolor{textGray}{gray}{.45}

% macro for table rules
\newcommand{\programCRule}[1]{%
  \cline{#1}%
}
\newcommand{\programHRule}[1]{%
  \cline{2-#1}%
}

% macro for empty session slots
\newcommand{\bookableSpace}{
  & \emph{\textcolor{textGray}{Bookable Space}}%
}

% diamond symbol for shortened titles
\newcommand{\diamondSymbol}{%
  \textsuperscript{%
    \diamond%
  }%
}

% speaker affiliation
\newcommand{\speakerAffiliation}[1]{%
  (#1)%
}

% lightning talk (title and author)
\newcommand{\lightningTalk}[2]{%
  \item \emph{#2:} #1%
}

% macro for no-video icon
\newcommand{\noVideo}{%
  \raisebox{-0.2\height}{%
    \includegraphics[height=8pt]{novideo.pdf}%
  }%
}
